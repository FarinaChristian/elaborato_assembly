% Inizia documento "articolo" con 
% margini per foglio A4, 11 punti
\documentclass[a4paper,11pt]{article}


% -- ESTENSIONI
\usepackage[T1]{fontenc}    % Codifica di caratteri T1
\usepackage[utf8]{inputenc} % Permette di usare caratteri UTF8
\usepackage[italian]{babel} % Imposta lingua italiana

% Permetti a latex di leggere le variabili d'ambiente
\usepackage{catchfile}

% Permette syntax highlighting del codice
\usepackage[outputdir=../docs_build]{minted}

% TOC con collegamenti ipertestuali
\usepackage{hyperref}

% Importa file .tex
\usepackage{subfiles}


% -- DEFINIZIONE COMANDI

% Definisci comando "getenv" per ottenere il valore delle variabili d'ambiente
% e leggere gli autori del progetto
\newcommand{\getenv}[2][]{%
  \CatchFileEdef{\temp}{"|kpsewhich --var-value #2"}{\endlinechar=-1}%
  \if\relax\detokenize{#1}\relax\temp\else\let#1\temp\fi}


% -- PREAMBOLO

% titolo in grassetto
\title{
	\textbf{
		Elaborato Assembly \\ 
		\noindent\rule{6cm}{0.4pt} \\  % riga orizzontale intermedia
		Architettura degli Elaboratori
	}
}

% ottieni gli autori dalle variabili d'ambiente
\author{
	\getenv{asm_rel_author1} \\
	\getenv{asm_rel_author2} \\	
	\getenv{asm_rel_author3}
}

% definisci data di scrittura
\date{Anno 2020/2021}


% -- INIZIO DOCUMENTO

\begin{document}

% - PAGINA CON TITOLO

% togli il numero di pagina dalla prima pagina
\pagenumbering{gobble}

% visualizza titolo, autori e data
\null  % Empty line
\nointerlineskip  % No skip for prev line
\vfill
\let\snewpage \newpage
\let\newpage \relax
\maketitle
\let \newpage \snewpage
\vfill 
\break % page break

% vai nella nuova pagina
\newpage
% riattiva la numerazione delle pagine
\pagenumbering{arabic}


% - PAGINA CON INDICE
\tableofcontents
\newpage


% - PAGINA CON INTRODUZIONE
\section{Introduzione}
\subfile{sections/1_intro}
\newpage


% - PAGINA CON INFO RELATIVE ALLE VARIABILI
\section{Variabili}
\subfile{sections/2_variables}
\newpage


% - PAGINA CON INFO RELATIVE ALLE FUNZIONI
\section{Funzioni}
\subfile{sections/3_functions}
\newpage


% - PAGINA CON IL DIAGRAMMA DI FLUSSO OPPURE DEL PSEUDO-CODICE
\section{Diagramma di flusso o pseudo-codice}
\subfile{sections/4_code}
\newpage


% - PAGINA CON LE SCELTE PROGETTUALI
\section{Scelte progettuali}
\subfile{sections/5_design_choices}
\newpage


\end{document}
